
\section{\texorpdfstring{\boldmath${\beta}$-Reduction}{Beta Reduction}}
Before we discuss \br, we first treat bound and free variables. Let us consider the following two functions: $\la x. x$ and $\la y. y$. These functions do the same.
($\la x. x) z$ returns the same value as ($\la y.y) z$, namely the term $z$. Therefore the name of the variables $x$ and $y$ do not matter. These variables are called bound variables.
A variable is considered a bound variable if it occurs in an abstraction and the variable name is a parameter in that abstraction. A variable is free if/iff?? it is not bound.
In the abstraction ($\la x. xy) z$, the variable $x$ is bound, while $y$ and $z$ are free.

With the definition of free variables, we can define substitution. Let us consider the last example ($\la x. xy) z$ again. After reducing this \lterm, all occurrences of the variable $x$ in the body of the function are be replaced by the variable $z$.
We denote this as follows: $(xy) \subst{x}{z} = zy$. The function application $(\la x. (\la x. x x) x)y$ should reduce to $(\la x. x x) y$ and $((\la x. x x) x) \subst{x}{y} = (\la x. x x) y$. So only the free variables in the body of the function are replaced.

More formally, the substitution of $x$ by $y$ in lambda term $M$ is defined by replacing all free occurences of $x$ by $y$.
\br is the process of replacing the function's parameter with the given argument. In the previous example, we had that $(\la x. x)y$ reduces to $y$.
%rewrite this sentence
A function with body $M$ and parameter $x$, $\la x. M$, applied to argument $N$, will yield $M \subst{x}{N}$
The rule below ($\beta$) formally defines. \br

\vspace{10pt}
\[
	(\la x. M)N \rightarrow M \subst{x}{N} \quad \quad (\beta)
\]

\vspace{10pt}
Any subterm of the form $(\la x. M) N$ is called a \bre. \bres are the terms that are reduced when \br is applied on that term.
% We also need rules to apply beta reduction on subterms. 
However, we do not have any rules that define that we can apply \br on subterms. We can now only apply it to the most outer term.
To solve this, we define the rule $\rightarrow_\beta$ which is the compatible closure of $\beta$.
We say that $\rightarrow_\beta$ is closed under $\mu$, $\nu$ and $\xi$, which are given below. That is, with $\rightarrow_\beta$, we can use $\beta$ with these rules.
The rules below will be followed by an example.

\vspace{-10pt}
\begin{center}
	\begin{tabularx}{\textwidth}{YYY}
		\begin{prooftree}
			\AxiomC{$M N \rightarrow M' N$} % Conclusion
			\RightLabel{$(\mu)$}
			\UnaryInfC{$M \rightarrow M'$}
		\end{prooftree}
		 & \quad
		\begin{prooftree}
			\AxiomC{$M N \rightarrow M N'$} % Conclusion
			\RightLabel{$(\nu)$}
			\UnaryInfC{$N \rightarrow N'$}
		\end{prooftree}
		 & \quad
		\begin{prooftree}
			\AxiomC{$\lambda x. M \rightarrow \lambda x. M'$} % Conclusion
			\RightLabel{$(\xi)$}
			\UnaryInfC{$M \rightarrow M'$}
		\end{prooftree}
	\end{tabularx}
\end{center}

\vspace{10pt}
\begin{prooftree}
	\AxiomC{$\mathbf{I}x \rightarrow_\beta \mathbf{I}$}
	\RightLabel{$(\xi)$}
	\UnaryInfC{$\la x. \mathbf{I}x \rightarrow_\beta \mathbf{I}$}
	\RightLabel{$(\nu)$}
	\UnaryInfC{$\mathbf{I}(\la x. \mathbf{I}x) \rightarrow_\beta \mathbf{II}$}
	\RightLabel{$(\mu)$}
	\UnaryInfC{$(\mathbf{I}(\la x. \mathbf{I}x))(\mathbf{II}) \rightarrow_\beta (\mathbf{II})(\mathbf{II})$}
\end{prooftree}
