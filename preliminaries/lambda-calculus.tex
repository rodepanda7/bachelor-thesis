% SOME REMARKS:
% NEVER START A SENTENCE WITH MATH. THE TERM .... DOES THIS
% MAKE BOUND VARIABLES MORE FORMAL AND PRECISE
% IN HET BEGIN OOK UITLEGGEN FREE VARIABLES

% FIRST DEFINE SUBSTITUTION FORMALLY AND FREE AND BOUND VARIABLES.
% GIVEN TERM M, WE WRITE M \SUBST X N. 
% EXPLAIN IN EXAMPLE
% DO NOT SUBSTITUTE X BY Y, BUT A TERM BY ANOTHER TERM

%ALTIJD EEN DING PER ZIN SCHRIJVEN
% DBLP VOOR PAPERS GEBRUIKEN
% use more definitions and theorem environments latex
% benoem nog een duidelijk voorbeeld voor value

\chapter{Lambda Calculus}
This section provides a detailed and formal description of the \lc. We define a formal grammar of the \lc and give examples of some terms in the \lc.
Then we explain \br, after which we define call-by-name evaluation and call-by-value evaluation.

% In this section, we will give a brief introduction of the lambda calculus.
% There are two basic lambda calculi: call by name lambda calculus and call by value lambda calculus, both formalised by Plotkin.
% The formalisation of those two lambda calculi will be given here and explained, since these form the basis for the rest of this research.
% We will first explain the grammer of the lambda calculus. Then we give beta reduction rules and we will define compatible closure, which is necessary to be able to apply the reduction rules everywhere on the expression.

\section{\texorpdfstring{Introduction to the \boldmath$\lambda$-calculus}{Introduction to the Lambda Calculus}}
The simplest and most basic grammar of the lambda calculus is as follows:

\vspace{10pt}
\begin{grammar}{
		\pr{$M, N, P, Q$}{$x \gors \lambda x.M \gors M N$}
	}
\end{grammar}

\vspace{10pt}
So, terms, denoted by $M, N, P$ or $Q$, can either be of the form $x$, $\lambda x.M$ or $M N$:
\begin{itemize}[noitemsep]
	\item $x$ is a variable, which is a symbol that represents an input or a value.
	\item $\lambda x.M$ is an abstraction. An abstraction is an anonymous function, where $x$ is the parameter and $M$ is the body of the function.
	\item $M N$ is a function application, where $M$ and $N$ are terms.
\end{itemize}

Now, we give an example of an abstraction and of a function application. $\lambda x. x$ is an abstraction. This specific abstraction is called the identity function and has one input parameter, namely $x$, and returns the input $x$.
This function is often abbreviated as $\mathbf{I}$. $(\lambda x. x) y$ is a function application. The anonymous function is applied to the variable $y$ and the \lterm reduces to the variable $y$.

\section{\texorpdfstring{\boldmath${\beta}$-Reduction}{Beta Reduction}}
Before we discuss \br, we first treat bound and free variables. Let us consider the following two functions: $\la x. x$ and $\la y. y$. These functions do the same.
($\la x. x) z$ returns the same value as ($\la y.y) z$, namely the term $z$. Therefore the name of the variables $x$ and $y$ do not matter. These variables are called bound variables.
A variable is considered a bound variable if it occurs in an abstraction and the variable name is a parameter in that abstraction. A variable is free if/iff?? it is not bound.
In the abstraction ($\la x. xy) z$, the variable $x$ is bound, while $y$ and $z$ are free.

With the definition of free variables, we can define substitution. Let us consider the last example ($\la x. xy) z$ again. After reducing this \lterm, all occurrences of the variable $x$ in the body of the function are be replaced by the variable $z$.
We denote this as follows: $(xy) \subst{x}{z} = zy$. The function application $(\la x. (\la x. x x) x)y$ should reduce to $(\la x. x x) y$ and $((\la x. x x) x) \subst{x}{y} = (\la x. x x) y$. So only the free variables in the body of the function are replaced.

More formally, the substitution of $x$ by $y$ in lambda term $M$ is defined by replacing all free occurences of $x$ by $y$.
\br is the process of replacing the function's parameter with the given argument. In the previous example, we had that $(\la x. x)y$ reduces to $y$.
%rewrite this sentence
A function with body $M$ and parameter $x$, $\la x. M$, applied to argument $N$, will yield $M \subst{x}{N}$
The rule below ($\beta$) formally defines. \br

\vspace{10pt}
\[
	(\la x. M)N \rightarrow M \subst{x}{N} \quad \quad (\beta)
\]

\vspace{10pt}
Any subterm of the form $(\la x. M) N$ is called a \bre. \bres are the terms that are reduced when \br is applied on that term.
% We also need rules to apply beta reduction on subterms. 
However, we do not have any rules that define that we can apply \br on subterms. We can now only apply it to the most outer term.
To solve this, we define the rule $\rightarrow_\beta$ which is the compatible closure of $\beta$.
We say that $\rightarrow_\beta$ is closed under $\mu$, $\nu$ and $\xi$, which are given below. That is, with $\rightarrow_\beta$, we can use $\beta$ with these rules.
The rules below will be followed by an example.

\vspace{-10pt}
\begin{center}
	\begin{tabularx}{\textwidth}{YYY}
		\begin{prooftree}
			\AxiomC{$M N \rightarrow M' N$} % Conclusion
			\RightLabel{$(\mu)$}
			\UnaryInfC{$M \rightarrow M'$}
		\end{prooftree}
		 & \quad
		\begin{prooftree}
			\AxiomC{$M N \rightarrow M N'$} % Conclusion
			\RightLabel{$(\nu)$}
			\UnaryInfC{$N \rightarrow N'$}
		\end{prooftree}
		 & \quad
		\begin{prooftree}
			\AxiomC{$\lambda x. M \rightarrow \lambda x. M'$} % Conclusion
			\RightLabel{$(\xi)$}
			\UnaryInfC{$M \rightarrow M'$}
		\end{prooftree}
	\end{tabularx}
\end{center}

\vspace{10pt}
\begin{prooftree}
	\AxiomC{$\mathbf{I}x \rightarrow_\beta \mathbf{I}$}
	\RightLabel{$(\xi)$}
	\UnaryInfC{$\la x. \mathbf{I}x \rightarrow_\beta \mathbf{I}$}
	\RightLabel{$(\nu)$}
	\UnaryInfC{$\mathbf{I}(\la x. \mathbf{I}x) \rightarrow_\beta \mathbf{II}$}
	\RightLabel{$(\mu)$}
	\UnaryInfC{$(\mathbf{I}(\la x. \mathbf{I}x))(\mathbf{II}) \rightarrow_\beta (\mathbf{II})(\mathbf{II})$}
\end{prooftree}

\section{Call-by-name and call-by-value}
Before we discuss call-by-name and call-by-value, from now called cbn and cbv, we first explain the concept of values.
In the \lc, we can distinguish between terms that are values and terms that are not a value.
Values are all terms that are a variable or an abstraction (function).
$x$ and $\la x. x$ are values, while $(\la x. x)z$ and $(\la x. xx) (\la x. xx)$ are not.

The letters $V$ and $W$ are used for values.
Terms, indicated with the letter $M$, $N$, $P$ or $Q$, can be any kind of term.
With this, we can define the call-by-name \lc and call-by-value \lc, $\la_n$ and $\la_v$ respectively.
Beta reduction of $\la_n$ and $\la_v$ is indicated by $\beta_n$ and $\beta_v$ respectively.
$\beta_n$ and $\beta_v$ are defined as follows:

\vspace{10pt}
\begin{tabular}{c c}
	$\beta_n: \quad (\lambda x.M) N \rightarrow M\subst{x}{N}$ &
	$\quad \beta_v: \quad (\lambda x.M) V \rightarrow M\subst{x}{V}$
\end{tabular}
%add description

These reductions are quite straightforward. With $\beta_n$, we replace the variable in the function by the argument without necessarily evaluating it.
With $\beta_v$, the argument should be a value, indicated by $V$. Only if the argument is a value, the variable of the function will be replaced by the argument.
We define $M\subst{x}{N}$ as the term M where all free occurrences of x are replaced by $N$.
\vspace{10pt}
Many \lterms can be reduced in multiple ways. $(\la x. x)(\la x. ((\la y. y) x))$, for example, has two reductions.
In each \br step, we also indicate the \bre on which the \br is applied by underlining that subterm. The two reductions are as follows:

\[
	\begin{array}{c c}
		\parbox[c]{0.5\textwidth}{\centering \textbf{Reduction 1}} & \parbox[c]{0.5\textwidth}{\centering \textbf{Reduction 2}} \\
		\vspace{-10pt}                                                                                                          \\
		\begin{aligned}
			\underline{(\lambda x. x)(\lambda x. ((\lambda y. y) x))} & \rightarrow \\
			\lambda x. \underline{((\lambda y. y) x)}                 & \rightarrow \\
			\lambda x. x
		\end{aligned}
		                                                           &
		\begin{aligned}
			(\lambda x. x)\underline{(\lambda x. ((\lambda y. y) x))} & \rightarrow \\
			\underline{(\lambda x. x)(\lambda x. x)}                  & \rightarrow \\
			\lambda x. x
		\end{aligned}
	\end{array}
\]

\vspace{10pt}
In reduction 1, the argument of the function is not reduced before applying the argument to the function, while the argument of the function in reduction 2 is reduced before applying the function to the argument.
Reduction 1 is related to call-by-name evaluation.

\vspace{10pt}

Now consider the following lambda term:

\vspace{10pt}
$(\lambda x. x + ((\lambda y.y)7)) ((\lambda x.x)30)$

\vspace{10pt}
If we do not define extra closure rules, we can only beta reduce this term with $\beta_n$, since $(\lambda x. x) 30$ is not a value.
It would be nice if we could reduce $(\lambda x. x) 30$ or $(\lambda x. x + ((\lambda y.y)7))$ in the bigger term.
This is made possible with the following closure rules. The rules are specified with a horizontal line.
If we can proof that everything above the line holds, then we can conclude everything below the line.

