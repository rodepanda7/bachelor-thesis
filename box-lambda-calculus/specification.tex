\section{\texorpdfstring{Speficitation of \boldmath${\la_b}$}{Specification of the call-by-box lambda calculus}}

% Curry Howard isomorphism states that the simply typed lambda calculus is equivalent to the intuitionistic propositional logic.
% Boxed lambda calculus is a fragment of the intuitionistic modal logic IS4.
% Boxed Values: Unlike in standard lambda calculus, boxes are not treated as values themselves but as containers for delayed computations.


% Here is the intuition from chat gpt:

% The boxed lambda calculus introduced in this paper, denoted as 
% 𝜆
% 𝑏
% λ 
% b
% ​
%  , emerges as a refined target for the modal embeddings of call-by-name (cbn) and call-by-value (cbv) lambda calculi into modal logic IS4. 
% Its intuition is centered around unifying these two paradigms through a single intermediate calculus that is "indifferent" 
% to the differences between cbn and cbv. Here’s a summary of the key ideas:

% Origin and Motivation
% The boxed lambda calculus is proposed as a refinement of the simpler modal target 
% 𝜆
% 2
% λ 
% 2
% ​
%  , which itself is a fragment of the intuitionistic modal logic IS4. This refinement addresses the following issues:

% Administrative Overhead: In the simpler 
% 𝜆
% 2
% λ 
% 2
% ​
%   target, both Girard's and Gödel's translations lead to numerous administrative reductions (like unnecessary unpacking and repacking of boxes) 
% that obscure the true computational steps. The boxed lambda calculus aims to reduce these steps.

% Standardization and Indifference: The authors wanted a target that unifies the two calling paradigms in a way that supports standardization theorems, 
% ensuring that both cbn and cbv can be extracted from a single, more general reduction system.

% Protecting-by-a-Box: The boxed lambda calculus introduces a new compilation technique, "protecting-by-a-box," 
% which improves on the classical "protecting-by-a-lambda" strategy. This approach efficiently captures both cbn and cbv by treating boxed values as a distinct syntactic category, isolating administrative steps and preserving the semantics of the source languages.

% Core Features of 
% 𝜆
% 𝑏
% λ 
% b
% ​
 
% Boxed Values: Unlike in standard lambda calculus, boxes are not treated as values themselves but as containers for delayed computations.

% Reduction Strategy: It adopts a new "call-by-box" (cbb) reduction paradigm, where function applications only reduce when the argument is a box, 
% effectively unifying cbn and cbv under a single reduction rule.

% Standardization: The calculus supports a refined standardization theorem, providing a uniform framework for reasoning about program evaluation across calling paradigms.

% Impact
% By refining the modal target to 
% 𝜆
% 𝑏
% λ 
% b
% ​
%  , the authors achieve a stronger form of the preservation and reflection properties, 
% enabling a more precise correspondence between source and target reductions. 
% This leads to a unified, modality-based understanding of cbn and cbv, bridging the gap between these foundational functional programming paradigms.



The terms of \lab are given by:

\begin{grammar}{
  \pr{$M, N, P, Q$}{$\boxe{x} \gors \lambda x.M \gors M N \gors \boxi{N}$}
}
\end{grammar}

\noindent Types are given by:

\begin{grammar}{
  \pr{$A, B$}{$X \gors A \fa B \gors \square A$}
}
\end{grammar}

\noindent The typing rules of \lab are give by:

\begin{prooftree}
  \def\extraVskip{5pt}
  \AxiomC{\phantom{A}}
  \RightLabel{Id$_\square$}
  \UnaryInfC{$\judgement{\Gamma, x : \boxt{A}}{\boxe{x} : A}$} % Conclusion
\end{prooftree}

\begin{prooftree}
  \def\extraVskip{5pt}
  \AxiomC{$\judgement{\Gamma, x : A}{M : B}$}
  \RightLabel{\impliesi}
  \UnaryInfC{$\judgement{\Gamma}{\lamb{x}{M} : A \fa B}$} % Conclusion
\end{prooftree}

\begin{prooftree}
  \def\extraVskip{5pt}
  \AxiomC{$\judgement{\Gamma}{M : A \fa B}$}
  \AxiomC{$\judgement{\Delta}{N : A}$}
  \RightLabel{\impliese}
  \BinaryInfC{$\judgement{\Gamma, \Delta}{M N : B}$} % Conclusion
\end{prooftree}

\begin{prooftree}
  \def\extraVskip{5pt}
  \AxiomC{$\judgement{\Gamma}{M : A}$}
  \RightLabel{$\square$-I}
  \UnaryInfC{$\judgement{\Gamma}{\boxi{M} : \boxt{A}}$} % Conclusion
\end{prooftree}

Since variables always occur inside $\varepsilon$, we need to redefine free and bound variables and substitution.
For \lab, we have: 

\begin{align*}
	& FV_b(\boxe{x})    & &= \{\boxe{x}\}                   & & BV(\boxe{x})    & &= \emptyset \\
	& FV_b(\lamb{x}{M}) & &= FV(M) \backslash \{\boxe{x}\}  & & BV(\lamb{x}{M}) & &= BV(M) \cup \{\boxe{x}\} \\
	& FV_b(M N)         & &= FV(M) \cup FV(N)               & & BV(M N)         & &= BV(M) \cup BV(N)
\end{align*}

Thus, for substitution, we get:

\[
\begin{aligned}
	& \boxe{x} \subst{\boxe{x}}{Q}            & &= Q \\
	& \boxe{y}  \subst{\boxe{x}}{Q}           & &= \boxe{y} \text{ if } (x \neq y)\\
	& (M N) \subst{\boxe{x}}{Q}               & &= M \subst{x}{Q} N \subst{x}{Q} \\
	& (\lamb{x}{M}) \subst{\boxe{x}}{Q}       & &= \lamb{x}{M} \\
	& (\lamb{y}{M}) \subst{\boxe{x}}{Q}       & &= \lamb{z}{M \subst{\boxe{y}}{\boxe{z}}\subst{\boxe{x}}{Q}} \text{ if } (x \neq y) \\
\end{aligned}
\]

with $z$ a variable defined by:
\begin{enumerate}
	\item%1
	If $\boxe{x} \notin FV_b(N)$ or $\boxe{y} \notin FV_b(M)$ then $z = y$
	\item%2
	Otherwise, $z$ can be any variable such that $\boxe{z} \notin FV(N)$ or $\boxe{z} \notin FV(M)$
\end{enumerate}

We prove by structural induction on the term $M$ that this substitution preserves well-typedness.  *? is well-defined ?*
So we prove: 
\[
  \judgement{\Gamma, x : A}{M : B} \mbox{ and } \judgement{\Gamma}{Q : A} \mbox{ implie } \judgement{\Gamma}{M \subst{\boxe{x}}{Q} : B}
\]

\begin{enumerate}
  \item%1 
  \[
  \boxe{x}[\boxe{x} := Q] = Q
  \]
  
  Since $Q$ is well-typed, and $Q$ is the result of the substitution, we have that well-typedness is preserved.
  \item%2
  \[
  \boxe{y}[\boxe{x} := Q] = \boxe{y} \quad \text{if} \ x \neq y
  \]

  We have that $x \neq y$, and the result of the substitution is $\boxe{y}$, which is well-typed.
  
  \item%3
  \[
  (M N)[\boxe{x} := Q] = (M[\boxe{x} := Q]) \ (N[\boxe{x} := Q])
  \]

  If we have that $M N$ is well-typed, then $M$ and $N$ are well-typed.
  By the inductive hypothesis, we get that $M[\boxe{x} := Q]$ is well-typed, and $N[\boxe{x} := Q]$ is well-typed.
  We also have that application of two well-typed terms is well-typed, so we have that $(M N)[\boxe{x} := Q]$ is well-typed.

  \item%4
  \[
  (\lamb{x}{M})[\boxe{x} := Q] = \lamb{x}{M}
  \]

  Since $\boxe{x}$ is a bound variable in $\lamb{x}{M}$, no substitution occurs. The result of the substitution is the well-typed term $\lamb{x}{M}$.

  \item%5
  \[
  (\lamb{y}{M})[\boxe{x} := Q] = \lamb{z}{(M[\boxe{y} := \boxe{z}][\boxe{x} := Q])} \quad \text{if} \ x \neq y
  \]

  If we have that $\lamb{y}{M}$ is well-typed, then $M$ is well-typed.
  By the inductive hypothesis, we have that $M[\boxe{y} := \boxe{z}]$ and $M[\boxe{y} := \boxe{z}][\boxe{x} := Q]$ are well-typed.
  We also havee that abstraction of a well-typed term is well-typed, so we have that \\
  $\lamb{z}{(M[\boxe{y} := \boxe{z}][\boxe{x} := Q])}$ is well-typed.
\end{enumerate}

The reduction rule we can use for this \lc is:

\begin{align*}
  (\lamb{x}{M})\boxi{N} \red M \subst{\boxe{x}}{N} \quad \quad (\bb)
\end{align*}

We define two relations: $\redbn$ and $\redbv$. The relation $\redbn$ will be used for cbn evaluation for embeddings of \lan in \lab.
The relation $\redbv$ will be used for cbn evalluation for embeddings of \lav in \lab. 

\begin{table}[htbp]
	\centering
	\begin{tabularx}{\linewidth}{|X|XXXXXX|}
		\hline
		\multicolumn{7}{|c|}{\textbf{Closure Rules}}\\
		\hline
		       & {$\beta$} & $\mu$        & $\mu_>$     & $\nu$        & $\nu_<$      & $\xi$ \\
		\hline
		$\rednb$   & $\bn$ & $\checkmark$ & -           & $\checkmark$ & $\checkmark$ & $\checkmark$ \\
		\hline
		$\redn$    & $\bn$ & $\checkmark$ & -           & 						 &              & \\
		\hline
		$\redw$    & $\bn$ & $\checkmark$ & -           & $\checkmark$ &              & \\
		\hline
    $\redvb$   & $\bv$ & $\checkmark$ & -           & $\checkmark$ & $\checkmark$ & $\checkmark$ \\
    \hline
    $\redv$    & $\bv$ & $\checkmark$ & -           & 						 & $\checkmark$ & \\
    \hline
    $\redbbox$ & $\bb$ & $\checkmark$ & $\checkmark$ & $\checkmark$ & $\checkmark$ & $\checkmark$ \\
    \hline
    $\redwe$   & $\bb$ & $\checkmark$ & $\checkmark$ & $\checkmark$ &              & \\
    \hline
    $\redweg$  & $\bb$ &              & $\checkmark$ & $\checkmark$ &              & \\
    \hline
    $\redwes$  & $\bb$ & $\checkmark$ &              &              & $\checkmark$ & \\
    \hline
	\end{tabularx}
	\end{table}

% \begin{center}
% 	\begin{tabularx}{\textwidth}{YY}
% 		\begin{prooftree}
% 			\AxiomC{$M \rightarrow M'$} % Conclusion
% 			\RightLabel{$(\mu)$}
% 			\UnaryInfC{$M N \rightarrow M' N$}
% 		\end{prooftree}
%     & \quad 
%     \begin{prooftree}
% 			\AxiomC{$M \rightarrow M'$} % Conclusion
% 			\RightLabel{$(\mu_>)$}
% 			\UnaryInfC{$M B \rightarrow M' B$}
% 		\end{prooftree}
% 	\end{tabularx}
%   \begin{tabularx}{\textwidth}{YYY}
% 		\begin{prooftree}
% 			\AxiomC{$N \rightarrow N'$} % Conclusion
% 			\RightLabel{$(\nu)$}
% 			\UnaryInfC{$M N \rightarrow M N'$}
% 		\end{prooftree}
%     & \quad 
% 		\begin{prooftree}
% 			\AxiomC{$N \rightarrow N'$} % Conclusion
% 			\RightLabel{$(\nu_<)$}
% 			\UnaryInfC{$V N \rightarrow V N'$}
% 		\end{prooftree}
%     & \quad 
%     \begin{prooftree}
% 			\AxiomC{$M \rightarrow M'$} % Conclusion
% 			\RightLabel{$(\xi)$}
% 			\UnaryInfC{$\lamb{x}{M} \rightarrow \lamb{x}M'$}
% 		\end{prooftree}
% 	\end{tabularx}
% \end{center}
