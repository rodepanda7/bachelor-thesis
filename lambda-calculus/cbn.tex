
\section{Call-by-name calculus}
In this section, we discuss the call-by-name \lc. 
The reduction rule that is used in this calculus is the same as the $\beta$ reduction rule.
However, we name the rule $\bn$ to make it clear that it is the reduction rule of the call-by-name \lc.

\begin{align*}
	(\lamb{x}{M})N \red M \subst{x}{N} \quad \quad \bn
\end{align*}

We can define weak reduction, $\redw$, as $\bn$ closed under $\mu$ and $\nu$. So $\redw$ has the only restiction that it cannot reduce under $\la$'s.
The relation $\redn$ is defined as $\bn$ closed under $\mu$. So using name evaluation, we can not evaluate the argument of a function. 
Call-by-name evaluation is defined as $\redns$. In the following example, we give the call-by-name evaluation of the \lterm $(\bi(\lamb{x}{\bi x}))(\bi \bi)$.
The \bre that is reduced is underlined.
Note that $\redn$ is deterministic, so there is only one way to apply $\redn$ on a \lterm.

\begin{eqnarray*}
	(\underline{\bi(\lamb{x}{\bi x})})(\bi \bi) &\redn& \underline{(\lamb{x}{\bi x}) (\bi \bi)} \\
	&\redn& \underline{\bi (\bi \bi)} \\
	&\redn& \underline{\bi \bi} \\
	&\redn& \bi 
\end{eqnarray*}
