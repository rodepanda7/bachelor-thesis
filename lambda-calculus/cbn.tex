
\section{Call-by-name calculus}
In this section, we discuss the call-by-name \lc. 
The reduction rule that is used in this calculus is the same as the $\beta$ reduction rule.
However, we name the rule $\bn$ to make it clear that it is the reduction rule of the call-by-name \lc.

\begin{definition}[Call-by-name reduction]
	\label{def:call-by-name-reduction}
	Let $M$ and $N$ be terms in the \lc. The relation $\redn$ is defined as follows:
	
	\begin{align*}
		(\lamb{x}{M})N \red M \subst{x}{N} \quad \quad \bn
	\end{align*}
\end{definition}

% DO WE NEED WEAK REDUCTION? IF NOT, REMOVE IT.
% EXPLAIN THE IDEA OF THE NAME REDUCTION

We can define weak reduction, $\redw$, as $\bn$ closed under $\mu$ and $\nu$. So $\redw$ has the only restiction that it cannot reduce under $\la$'s.
The relation $\redn$ is defined as $\bn$ closed under $\mu$. So using name evaluation, we can not evaluate the argument of a function. 
Call-by-name evaluation is defined as $\redns$. In the following example, we give the call-by-name evaluation of the \lterm $(\bi(\lamb{x}{\bi x}))(\bi \bi)$.
The \bre that is reduced is underlined.
Note that $\redn$ is deterministic, so there is only one way to apply $\redn$ on a \lterm.

\begin{example}
	This is an example of call-by-name evaluation in \lan.
	\label{ex:call-by-name}
	\begin{eqnarray*}
		(\underline{\bi(\lamb{x}{\bi x})})(\bi \bi) &\redn& \underline{(\lamb{x}{\bi x}) (\bi \bi)} \\
		&\redn& \underline{\bi (\bi \bi)} \\
		&\redn& \underline{\bi \bi} \\
		&\redn& \bi 
	\end{eqnarray*}
\end{example}


\begin{table}[htbp]
\centering
\begin{tabularx}{\linewidth}{|X|XXXX|}
  \hline
  \textbf{Relation} & $\boldsymbol{\beta}$ & $\boldsymbol{\mu}$ & $\boldsymbol{\nu}$ & $\boldsymbol{\xi}$ \\
	\hline
	$\rednb$ & $\bn$ & $\checkmark$ & $\checkmark$ & $\checkmark$ \\
	\hline
	$\redn$  & $\bn$ & $\checkmark$ & 						 &  \\
	\hline
	$\redw$  & $\bn$ & $\checkmark$ & $\checkmark$ &  \\
	\hline
\end{tabularx}
\caption{Closure Rules for Different Reduction Relations}
\label{tab1:closure-rules}
\end{table}
