\section{Call-by-value calculus}
Before we can define the call-by-value \lc (abbreviated as ), we first define what we mean by values. 
Values are all \lterms that are either a variable or an abstraction. Values are usually denoted by $V$ or $W$.   

\begin{definition}[Values]
	\label{def:values}
	Values $V$ are defined as follows:	
	\begin{align*}
		\begin{grammar}{
			\pr{$V$}{$x \gors \lambda x.M$}
		}
		\end{grammar}
	\end{align*}
\end{definition}

In the \lav, we can only reduce if the argument is a value. Therefore, the reduction rule is as follows:

\begin{definition}[Call-by-value reduction]
	Let $M$ and $V$ be terms in the \lc with $V$ a value. The relation $\redv$ is defined as follows:
	\begin{align*}
		(\lamb{x}{M})V \red M \subst{x}{V} \quad \quad \bv 
	\end{align*}
\end{definition}

In order to define the $\redv$ relation, we first define the rule $\nu_<$. 

\begin{definition}[$\nu_<$]
	\label{def:nu-less}
	Let $M$ and $V$ be terms in the \lc with $V$ a value. The relation $\nu_<$ is defined as follows:
	
	\begin{prooftree}
		\def\extraVskip{5pt}
		\AxiomC{$N \red N'$}
		\RightLabel{$\nu_<$}
		\UnaryInfC{$V N \red V N'$}
	\end{prooftree}
\end{definition}

The relation $\redv$ can be defined as $\bv$, closed under $\mu$ and $\nu_<$. 
The $\nu_<$ forces evaluation from left to right and $\bv$ makes sure that the argument needs to be a value. 
In the following example, we give the call-by-value evaluation of the \lterm $(\bi(\lamb{x}{\bi x}))(\bi \bi)$.
The \bre that is reduced is underlined.
Note that $\redv$ is deterministic, so there is only one way to apply $\redv$ on a \lterm.

% EXPLAIN THIS REDUCTION MORE. MAKE SURE THAT THE READER UNDERSTANDS THIS EXAMPLE.
\begin{example}
	This is an example of call-by-value evaluation in \lan.
	\label{ex:call-by-value}
	\begin{eqnarray*}
		(\underline{\bi(\lamb{x}{\bi x})})(\bi \bi) &\redv& (\lamb{x}{\bi x}) (\underline{\bi \bi}) \\
		&\redv& \underline{(\lamb{x}{\bi x}) \bi} \\
		&\redv& \underline{\bi \bi} \\
		&\redv& \bi 
	\end{eqnarray*}
\end{example}

\begin{table}[htbp]
\centering
\begin{tabularx}{\linewidth}{|X|XXXXX|}
	\hline
	\textbf{Relation} & $\boldsymbol{\beta}$ & $\boldsymbol{\mu}$ & $\boldsymbol{\nu}$ & $\boldsymbol{\nu_<}$ & $\boldsymbol{\xi}$ \\
		\hline
		$\rednb$ & $\bn$ & $\checkmark$ & $\checkmark$ & $\checkmark$ & $\checkmark$ \\
		\hline
		$\redn$  & $\bn$ & $\checkmark$ & 						 &  & \\
		\hline
		$\redw$  & $\bn$ & $\checkmark$ & $\checkmark$ & & \\
		\hline
    $\redvb$ & $\bv$ & $\checkmark$ & $\checkmark$ & $\checkmark$ & $\checkmark$ \\
    \hline
    $\redv$  & $\bv$ & $\checkmark$ & 						 & $\checkmark$ & \\
    \hline
\end{tabularx}
\caption{Closure Rules for Different Reduction Relations}
\label{tab2:closure-rules}
\end{table}
