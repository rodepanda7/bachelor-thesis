\section{\texorpdfstring{\boldmath${\beta}$-Reduction}{Beta Reduction}}
Before we discuss \br, we first define bound and free variables and substitution.

\subsection{Variables}
Let us first define free variables. 
Let $P$ be any term in the \lc and $FV(P)$ be the set that contains all free variables in $P$. We define $FV$ inductively on $P$.
Since $P$ can only be a variable, an abstraction or a function application, $P$ can only be of the form $x$, $\lamb{x}{M}$ and $M N$.
Therefore, we define $FV$ as follows:

\[
\begin{aligned}
	& FV(x)           & &= \{x\} \\
	& FV(\lamb{x}{M}) & &= FV(M) \backslash \{x\} \\
	& FV(M N)         & &= FV(M) \cup FV(N)
\end{aligned}
\]

The variable in a \lterm that solely consists of a variable is free. All free variables in an abstraction are those that are not the parameter of the abstraction.
For example, the \lterm $\lamb{x}{yx}$ has one free variable, $y$. We can use the following reasoning:

\begin{eqnarray*}
	FV(\lamb{x}{yx}) &=& FV(y x) \backslash \{x\} \\
	&=& (FV(x) \cup FV(y)) \backslash \{x\} \\
	&=& (\{x\} \cup \{y\}) \backslash \{x\} \\
	&=& \{x, y\} \backslash \{x\} \\
	&=& \{y\}
\end{eqnarray*}

Let us now define bound variables. Again, let $P$ be any term in the \lc and $BV(P)$ be the set of all bound variables in $P$. 
We define $BV$ inductively on $P$ as follows:

\[
\begin{aligned}
	& BV(x)           & &= \emptyset \\
	& BV(\lamb{x}{M}) & &= BV(M) \cup \{x\} \\
	& BV(M N)         & &= BV(M) \cup BV(N)
\end{aligned}
\]

We say that the abstraction $\lamb{x}{M}$ binds the variable $x$. Using our previous example of the \lterm $\lamb{x}{yx}$, we can now reason that $BV(\lamb{x}{xy}) = \{x\}$. 

\begin{eqnarray*}
	BV(\lamb{x}{xy}) &=& BV(xy) \cup \{x\} \\
	&=& (BV(x) \cup BV(y)) \cup \{x\} \\
	&=& (\emptyset \cup \emptyset) \cup \{x\} \\
	&=& \{x\}
\end{eqnarray*}

% MAYBE KEEP $\lamb{x}{(\lamb{x}{x})x}$ EXAMPLE

The following example will demonstrate that using two different variables with the same name in a \lterm is quite tricky: $(\lamb{x}{x})x$.
By the definitions of $FV$ and $BV$, we can reason that $FV((\lamb{x}{x})x) = \{x\}$ and $BV((\lamb{x}{x})x) = \{x\}$.
This is quite confusing and we prefer to write $(\lamb{y}{y})x$ instead, where $y$ is a new variable with a different name.
We now have $FV((\lamb{y}{y})x) = \{x\}$ and $BV((\lamb{y}{y})x) = \{y\}$.
These \lterms are equivalent by the notion of \emph{alpha equivalence}, which states that \lterms that differ only by the names of bound variables are considered the same.
We will discuss this in more detail after we define substitution.

\subsection{Substitution}
% EMPHASIZE THAT SUBSITUTING IS REPLACING FREE VARIABLES BY ANY LAMBDA TERM. OTHERWISE, RULE 3 IS CONFUSING.
% MAYBE ADD EXAMPLE OF SUBSTITUTION
Keeping the definitions of free and bound variables in mind, we now define substitution. 
In this research, substitution is often denoted as $P \subst{x}{Q}$ and defined inductively on $P$ by:

\[
\begin{aligned}
	& x \subst{x}{Q}             & &= Q \\
	& y \subst{x}{Q}             & &= y \text{ if } (x \neq y)\\
	& (M N) \subst{x}{Q}         & &= M \subst{x}{Q} N \subst{x}{Q} \\
	& (\lamb{x}{M}) \subst{x}{Q} & &= \lamb{x}{M} \\
	& (\lamb{y}{M}) \subst{x}{Q} & &= \lamb{z}{M \subst{y}{z}\subst{x}{Q}} \text{ if } (x \neq y) \\
\end{aligned}
\]


% ​SUBSTITUTION FOR INTRINSICALLY TYPED LAMBDA CALCULUS
  
% (x:A)\subst(x:A)(Q:A)
% (y:B)\subst(x:A)(Q:A)
% (MN)\subst(x:A)(Q:A)
% (λ(x:A).M)\subst(x:A)(Q:A)
% (λ(y:B).M)\subst(x:A)(Q:A)  
% =Q
% =y if (x
% 
% =y)
% =(M\subst(x:A)(Q:A))(N\subst(x:A)(Q:A))
% =λ(x:A).M
% =λ(z:B).(M\subst(y:B)(z:B)\subst(x:A)(Q:A)) if (x
% 
% =y)
% ​


with $z$ a variable defined by:
\begin{enumerate}
	\item%1
	If $x \notin FV(N)$ or $y \notin FV(M)$ then $z = y$
	\item%2
	Otherwise, $z$ can be any variable such that $z \notin FV(N)$ or $z \notin FV(M)$
\end{enumerate}

The variable '$z$' always exists. Choosing a fresh variable for $z$ will always satisfy condition two.
It might not be clear why need to use the variable $z$ and one can expect that the following rule is good enough:
\begin{align*}
	(\lamb{y}{M}) \subst{x}{Q} &= \lamb{z}{M \subst{x}{Q}} \text{ if } (x \neq y)
\end{align*}

However, the following example will illustrate why the more complicated rule is necessary.
Consider the substitution $(\lamb{y}{xy})\subst{x}{y}$. The free variable of \lterm $\lamb{y}{xy}$ is $x$ and the bound variable is $y$.
If we use the simple rule above, we get that $(\lamb{y}{xy})\subst{x}{y} = \lamb{y}{yy}$, the meaning of the \lterm changes. 
However, the free variable we substituted, is not free anymore. The variable $y$ is not a free variable in the term that we substituted $y$ in.
Therefore, we change the bound variable $y$ in $\lamb{y}{xy}$ to a new variable $z$. We can pick $z$ as $z$ is not a free variable in $xy$.
Following the more complex rule, we get: 
\begin{eqnarray*}
	(\lamb{y}{xy})\subst{x}{y} &=& \lamb{z}{(xy)\subst{y}{z}\subst{x}{y}} \\
	&=& \lamb{z}{yz}
\end{eqnarray*}

\subsection{\texorpdfstring{\boldmath${\alpha}$ equivalence}{Alpha equivalence}}
Here, we define \aeq. The idea is that renaming bound variables does not change the meaning of a \lterm.
For example, the \lterms $\lamb{x}{x}$ and $\lamb{y}{y}$ are equivalent ($\lamb{x}{x} \aeqr \lamb{y}{y}$), since renaming all bound variables can result in identical \lterms.
The \lterms $\lamb{x}{x}$ and $\lamb{y}{y}$ represent the same idea.
We can define the relation $P \aeqr P'$ inductively on $P$ and $P'$ as follows:

\[
\begin{aligned}
	& x \aeqr y                      & &  \text{ if } x = y \\
	& MN \aeqr M'N'                  & & \text{ if } M \aeqr M' \text{ and } N \aeqr N' \\
	& \lamb{x}{M} \aeqr \lamb{y}{M'} & & \text{ if } M \aeqr M'\subst{y}{x} \text{ with } x = y \text{ or } x \notin FV(M') \\
\end{aligned}
\]

\noindent The last rule states that abstractions $\lamb{x}{M}$ and $\lamb{y}{M'}$ are equivalent if $M$ and $M'$ are equivalent after substituting $y$ for $x$ in $M'$.
However, $x$ should be equal to $y$ or $x$ should not be a free variable in $M'$. 
\begin{itemize}
	\item 
	$\lamb{x}{zx}$ is equivalent to $\lamb{y}{zy}$, since $zy\subst{y}{x} = zx \aeqr zx$ and $x$ is not a free variable in $zy$.
	\item
	$\lamb{z}{zx}$ is not to $\lamb{y}{zy}$, since $z$ is a free variable in $zy$ and $x$ is not equal to $y$.
\end{itemize}


\subsection{\texorpdfstring{\boldmath${\beta}$-Reduction}{Beta Reduction}}
Now, we are ready to discuss \br. In the \lc, there is one way to simplify terms, which is \br. We can define:

\begin{align*}
	(\lamb{x}{M})N \red M \subst{x}{N} \quad \quad (\beta)
\end{align*}

So a \lterm of the form $(\lamb{x}{M})N$ can be reduced by substituting $x$ by $N$ in $M$. 
However, without additional rules, we cannot apply \br to subterms. 
For example, the following reduction would not be possible: $\lamb{x}{(\lamb{y}{y})x} \red \lamb{x}{x}$.
Therefore, we can define the following rules:

\vspace{-10pt}
\begin{center}
	\begin{tabularx}{\textwidth}{YYY}
		\begin{prooftree}
			\AxiomC{$M \rightarrow M'$} % Conclusion
			\RightLabel{$(\mu)$}
			\UnaryInfC{$M N \rightarrow M' N$}
		\end{prooftree}
		 & \quad
		\begin{prooftree}
			\AxiomC{$N \rightarrow N'$} % Conclusion
			\RightLabel{$(\nu)$}
			\UnaryInfC{$M N \rightarrow M N'$}
		\end{prooftree}
		 & \quad
		\begin{prooftree}
			\AxiomC{$M \rightarrow M'$} % Conclusion
			\RightLabel{$(\xi)$}
			\UnaryInfC{$\lambda x. M \rightarrow \lambda x. M'$}
		\end{prooftree}
	\end{tabularx}
\end{center}

Now, we define $\redb$ as $\beta$ closed under $\mu$, $\nu$ and $\xi$. 
So with $\redb$ we can use \br on all subterms if the subterm is of the form $(\lamb{x}{M})N$. A subterm of the form $(\lamb{x}{M})N$ is called a \bre.
In one reduction, we may need to use multiple rules. The \bre that is reduced is underlined.
For instance, we need to use the $\mu$ and $\xi$ rule in the last example of the following reductions: 

\begin{align*}
	(\underline{\bi \bi})(\bi \bi) 							&\redb \bi (\bi \bi) 							 & &\beta \text{ with } \mu \text{ rule } \\
	(\bi \bi)(\underline{\bi \bi}) 							&\redb (\bi \bi) \bi 							 & &\beta \text{ with } \nu \text{ rule } \\
	\lamb{x}{\underline{\bi x}} 								&\redb \lamb{x}{x} 								 & &\beta \text{ with } \xi \text{ rule } \\
	(\bi(\lamb{x}{\underline{\bi x}}))(\bi \bi) &\redb (\bi(\lamb{x}{x}))(\bi \bi) & &\beta \text{ with } \mu \text{ and } \xi \text{ rule } \\
\end{align*}
